\begin{frame}
	\frametitle{Procedimento 01}
		\only<1>{
			\framesubtitle{Características}
				\begin{itemize}
					\item MEL: 13 coeficientes.
					\item BARK: 24 coeficientes.
					\item Redimensionar o sinal.
					\item Determinar o máximo de transformações.
					\item Transformadas wavelet-packet até o nível máximo.
					\item Selecionar a melhor combinação \textit{wavelet}/escala.
				\end{itemize}
		}
		\only<2>{
			\framesubtitle{Comprimento e máximo de transformações do sinal}
			\par Determinação do tamanho ótimo para as transformações
			\begin{equation}
				tamanhoOtimo=2^{proxInt(\log_{2}tamanho)}
				\label{eq:optimalSize}
			\end{equation} 
			
			\par Quantidade máxima de transformações
			\begin{equation}
				maxTrans=\log_{2}(tamanho) \qquad.
				\label{eq:maxWaveletTransf}
			\end{equation}
		}
		\only<3>{
			\framesubtitle{Algoritmo}
			\input{codeListings/experiment01Algo.tex}
		}
\end{frame}

\begin{frame}
	\frametitle{Procedimento 02}
	\only<1>{
		\framesubtitle{Características}
		\begin{itemize}
			\item Banda crítica: BARK.
			\item Wavelet: Haar.
			\item Modelos de 10\% 20\% 30\% 40\% e 50\%.
			\item Sorteio aleatório de vetores de características.
			\item Verifica a acurácia e o EER de classificadores \textit{pattern-matching} por distâncias Euclidiana e Manhattan.
		\end{itemize}
	}
	\only<2>{
		\framesubtitle{Algoritmo}
		\input{codeListings/experiment02Algo.tex}
	}
\end{frame}

\begin{frame}
	\frametitle{Procedimento 03}
	\only<1>{
		\framesubtitle{Características}
		\begin{itemize}
			\item Banda crítica: BARK.
			\item Wavelet: Haar.
			\item Modelos de 10\% 20\% 30\% 40\% e 50\%.
			\item Sorteio aleatório de vetores de características.
			\item Verifica a acurácia e o EER de uma \textit{Support Vector Machine (SVM)}.
		\end{itemize}
	}
	\only<2>{
		\framesubtitle{Características da SVM}
		\scalebox{1.3}{
			\input{../monography/images/3layersSVMTikzPicture.tex}
		}
	}
	\only<3>{
		\framesubtitle{Características da SVM}
		\begin{itemize}
			\item Três camadas: Entrada, segunda com elementos ativos não-lineares de núcleos Gaussianos e saída; 
			\item Inexistem pesos entre a camada de entrada e a camada intermediária;
			\item A saída de cada elemento da camada intermediária conecta-se com o único elemento da camada de saída por meio dos pesos $p_0, p_1, .... p_{X-1}$;
			\item O valor de saída consiste na combinação linear dos pesos com os valores recebidos como entrada da camada de saída;
			\item Solução direta de um sistema linear quadrado, isto é, possível e determinado.
		\end{itemize}
		
		\par Todos os arranjos para a seleção dos vetores de treinamento e testes, assim como demais detalhes, são idênticos àqueles do procedimento 02.
	}
	\only<4>{
		\framesubtitle{Algoritmo}
		\input{codeListings/experiment03Algo.tex}
	}
\end{frame}