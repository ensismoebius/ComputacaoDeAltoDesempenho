\begin{frame}
	\frametitle{Estrutura da Estratégia Proposta}
		\only<1>{
			\framesubtitle{Diagrama}
			\begin{textblock*}{\linewidth}(0.3cm,2cm)
				\input{../monography/images/strategicStructure.tex}
			\end{textblock*}
		}
		\only<2>{
			\framesubtitle{Wavelets usadas}
				\begin{itemize}
					\item Haar.
					\item Beylkin com suporte 18.
					\item Vaidyanathan de suporte 24.
					\item Daubechies de	suportes 4 até 76.
					\item Symmlets com suportes 8, 16 e 32.
					\item Coiflets com suportes 6, 12, 18, 24 e 30.
				\end{itemize}
		}
		\only<3>{
			\framesubtitle{Métricas}
				\begin{itemize}
					\item Tabela de confusão.
					\begin{itemize}
						\item EER (Equal Error Rate).
						\item Acurácia e seu respectivo desvio padrão.
					\end{itemize}
				\end{itemize}
		}
		\only<4>{
			\framesubtitle{Tabela de confusão}
				\begin{itemize}
					\item \textbf{TP}: Quantidade de itens verdadeiros classificados como tal (\textit{True Positive}).
					\item \textbf{TN}: Quantidade de itens falsos classificados como tal (\textit{True Negative}).
					\item \textbf{FN}: Quantidade de itens verdadeiros classificados como falsos (\textit{False Negative}).
					\item \textbf{FP}: Quantidade de itens falsos classificados como verdadeiros (\textit{False Positive}).
				\end{itemize} 
				\input{../monography/tables/results/confusionMatrices/confusionMatrixSample.tex}
		}
		\only<5>{
			\framesubtitle{Acurácia e EER}
				\begin{columns}
					\column{.5\textwidth}
					\begin{equation}
						acuracia = \dfrac{TP + TN}{N} \qquad,
						\label{eq:calculoDaAcuracia}
					\end{equation}
					
					\column{.5\textwidth}
					\par São calculadas tabelas de confusão por um número suficiente de vezes até que \textbf{\textit{FAR} = \textit{FRR} = \textit{EER}}, a cada ciclo os vetores de características são comutados de forma aleatória.\newline

					\begin{equation}
						FAR=\dfrac{FP}{TN+FP} \qquad,
						\label{eq:FAR}
					\end{equation}
					
					\begin{equation}
						FRR=\dfrac{FN}{TP+FN} \qquad,
						\label{eq:FRR}
					\end{equation}
				\end{columns}
		}
\end{frame}